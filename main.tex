%!TEX program = pdflatex
%!TeX encoding = UTF-8
%!TeX spellcheck = en_US
\documentclass[12pt]{article}
\usepackage[a4paper,margin=1in]{geometry} % Control page layout, \newgeometry{} & \restoregeometry
\usepackage{pdflscape} % Landscape environment for landscape view of PDF instead of portrait
\usepackage{changepage} % wider abstract
\usepackage[title,titletoc,header]{appendix} % Appendices environment
%\def\appendixname{附录}
\usepackage{soul} % Spacing & underline: \ul -> underline

% References
\usepackage{hyperref}
\hypersetup{colorlinks=true,linkcolor=black,citecolor=black,urlcolor=black,unicode}
\usepackage[backend=biber,style=numeric,sorting=ynt,citestyle=numeric-comp,autocite=inline]{biblatex}
\addbibresource{ref.bib}

% Math
\usepackage{amsmath, amsfonts} % Basic math equations & fonts
\usepackage{mathrsfs} % Ralph Smith’s Formal Script, \mathscr{F}
\usepackage{siunitx} % SI units
%\SI{5.2}{\kilo\watt\hour}, \SIrange{9}{12}{\hour}
%\si{\kilo\gram}
\sisetup{range-phrase={-}}
\usepackage{isomath} % ISO math format
\let\mat\boldsymbol
%$\mat{A}$ print bold italic matrix A
\usepackage{physics} % Physics equations
%vectors: \vb{} for bold, \va{} for arrowed, \vu{} for unit, \norm{} for norm
%\abs{} for absolute value, \Pr() for probability, \var{x-x_0} for variation
%Differential: \dd[n]{x}, \dd(\cos[2](x))
%Derivative: \dv{x}, \dv[n]{f}{x}, \dv{t}(\grande e^x)
%Partial: \pdv{x}, \pdv[n]{f}{x}
%Cal3: \grad{\va{a}}, \div{\va{a}}, \curl{\va{a}}, \laplacian{\va{a}}
\usepackage{esvect} % Beautiful vector
%$\vv{AB}$ for vector AB
\usepackage{xfrac} % Inline fraction
%\sfrac{1}{2} for inline 1/2

% Tables
\usepackage{tabularx,array} % Tabularx environment in array environment for costomized tables, e.g. \multicolumn
\usepackage{booktabs} % For \hline in tabularx environment with \toprule \midrule \bottomrule, partial rules with \cmidrule(lr){2-3}
\usepackage{multicol,multirow} % For \multicolumn{cols}{pos}{text} & \multirow{number of rows}{width or *}{text} in tabularx environment
\usepackage{longtable} % For tables that span multiple pages
\usepackage{colortbl} % Use color in table rows

% Figures & Animations
\usepackage{float} % Force tables, figures, pseudocode, etc. to stay EXACTLY where there're inserted with H
\usepackage{graphicx} % Insert picture with \includegraphics{}
\usepackage{subfigure} % Insert subfigures with subfigure environment
\usepackage{animate} % Insert animations with \animategraphics[options]{frame rate}{file basename}{first}{last}

% Color
\usepackage{color,xcolor}
% Code color
\definecolor{vsblue}{RGB}{0,0,255} % Keywords
\definecolor{vsgreen}{RGB}{0,128,0} % string
\definecolor{vscomment}{RGB}{106,115,125} % Comments
\definecolor{vspurple}{RGB}{175,0,219} % True, False, None
\definecolor{vsorange}{RGB}{200,75,49} % int, str, etc.
\definecolor{vscyan}{RGB}{0,122,204} % __init__, self
\definecolor{vsoperator}{RGB}{215,58,73} % Operators
% Table color
\definecolor{lightblue}{HTML}{F8FCFF}
\definecolor{saturatedlightblue}{HTML}{F4FAFF}

% Code
\usepackage{algpseudocode}
\usepackage[vlined,ruled,linesnumbered]{algorithm2e}
\usepackage{algorithmicx}
\usepackage{listings}
\lstset{
	% set frame
	frame={shadowbox},
	frameround=tttt,
	backgroundcolor=\color{lightblue},
	rulecolor=\color{gray},
	numbers=left, % Line numbers
	numbersep=5pt, % Line number spacing
	stepnumber=1, % Line number every 4 lines
	% detect language
	language=Python,
	stringstyle=\color{vsgreen},
	numberstyle=\color{black},
	keywordstyle=\bfseries\color{vsblue},
	keywords=[2]{True, False, None}, keywordstyle=[2]\bfseries\color{vspurple},
	identifierstyle=\color{black!90}, % Variables/function names
	emph={__init__, self}, % Special identifiers
	emphstyle=\color{vscyan}, % Teal special identifiers
	morekeywords=[3]{int, str, list, tuple, set, dist}, keywordstyle=[3]\bfseries\color{vsorange},
	% text format
	columns=flexible,
	basicstyle=\ttfamily\small\color{black},
	commentstyle=\color{vscomment},
	tabsize=3,
	breaklines=true, % Enable link breaking
	breakatwhitespace=true, % Allow breaks at whitespace
	showstringspaces=false % Hide string space markers
}

% Header & footer
\usepackage{lastpage}
\usepackage{fancyhdr}
\pagestyle{fancy}
\fancyhf{}
\fancyhead[C]{Paper}
\fancyfoot[C]{\thepage}
\usepackage{caption} % Costomize caption
\captionsetup[figure]{font={small},
	labelfont=bf,
	labelsep=space,
	%name=,
	position=below,
	justification=centering}
\captionsetup[table]{font={small},
	labelfont=bf,
	labelsep=space,
	%name=,
	position=above,
	justification=centering}

\begin{document}
	\title{Temp}
	\maketitle
	\begin{adjustwidth}{-1cm}{-1cm}
		\begin{abstract}
			temp
			\\
			\noindent\textbf{Keywords: }
		\end{abstract}
	\end{adjustwidth}

	%\newpage
	%\tableofcontents

	\newpage
	\section{Introduction}

%	\begin{table}[H]
%		\caption{核心参数的符号说明}
%		\zihao{-4}
%		\centering
%		\renewcommand{\arraystretch}{0.8}
%		\begin{tabular}{>{\zihao{5}}l>{\zihao{5}}l>{\zihao{5}}l}
%			\hline
%			\rowcolor{lightblue} \textbf{符号} & \textbf{意义} & \textbf{单位} \\
%			\hline
%			\rowcolor{saturatedlightblue} $T_I/\mathbf{T}$ & 计算任务$T_I\in$任务集合$\mathbf{T}$,可分为绿电$T_G$或传统电能$T_T$供电 & \\
%			\rowcolor{lightblue} $j/\mathbf{J}$ & 分为高、中、低三种的任务优先级类型索引$j\in$类型集合$\mathbf{J}$ & \\
%			\rowcolor{saturatedlightblue} $t/\mathbf{H}$ & 时刻$t\in$\SI{24}{\hour}时间集合$\mathbf{H}$,$\mathbf{H}=\{0,1,\ldots,23\}$ & \si{\hour} \\
%			\rowcolor{lightblue} $w_i/\mathbf{W}$ & 第$i$个时间窗口$w_i\in$时间窗口集合$\mathbf{W}$ & \si{\hour} \\
%			\rowcolor{saturatedlightblue} $D_I$ & 处理每个任务$T_I$可容忍的延时 & \si{\hour} \\
%			\rowcolor{lightblue} $c_i/\mathbf{c}$ & 第$i$个服务器集群$c_i\in$服务器集群集合$\mathbf{c}$ & \\
%			\rowcolor{saturatedlightblue} $N_c$ & 计算中心所有服务器集群总数量 & 个 \\
%			\rowcolor{lightblue} $\epsilon$ & 鲁棒性函数误差变量 & \\
%			\rowcolor{saturatedlightblue} $\xi$ & 分布鲁棒约束的容忍阈值(允许的违约概率上限),$\xi=0.05$ & \\
%			\hline
%		\end{tabular}
%		\label{table: parameters definitions}
%	\end{table}


%	\begin{align}
%		\label{equation: submodel two LP}
%		& \min \sum_{t \in \mathbf{H}} \sum_{j \in \mathbf{J}} p_T(t)\cdot E_G(t)\nonumber \\
%		\text{s.t.}&
%		\begin{cases}
%			\begin{aligned}
%				&&\sum_{j \in \mathbf{J}} n_j(t) \cdot P_j &= e_{G}(t) + e_{T}(t),\,\forall\, t \in \mathbf{T}\\
%				&&0&\leq e_{G}(t)\leq E_G(t),\,\forall\, t \in \mathbf{H}\\
%				&&E_T(t)&\geq\sum_{j\in\mathbf{J}} n_j(t)\cdot P_j(t) - E_G(t),\,\forall\,t \in \mathbf{H}\\
%				&&\sum_{t \in \mathbf{H}} N_j(t) &= \sum_{t \in \mathbf{H}} n_j(t),\,\forall\, j\in \mathbf{J}\\
%				&&\delta x_j(T_I)&= 0,\,\forall\, T_I \text{ where } U(T_I) = 3
%			\end{aligned}
%		\end{cases}
%	\end{align}


%	\begin{figure}[H]
%		\centering
%		\includegraphics[width=0.6\linewidth]{figure/DtImpact.jpg}
%		\caption{$D_I$对总成本和违约概率的影响图。}
%		\label{fig: D_t's Impact}
%	\end{figure}

	\newpage
	\printbibliography[heading=bibintoc]

	\newpage
	\begin{appendices}
		\section{Temp}
	\end{appendices}
\end{document}